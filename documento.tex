

\include{prétextuais/inalteraveis/documentclass}
\input{pacotes}
% ---
% Informações de dados para CAPA e FOLHA DE ROSTO
% ---
\titulo{Projeto Integrador:\\ Modelo IEL com \abnTeX}
\autor{Cristian Robert Belão de Meira Junior\\Douglas Cristiano Morona\\Felipe Rodrigues\\Fernando Macedo Kramar\\Pedro Henrique Brunning}
\local{São José dos Pinhais}
\data{2021}
\orientador{Denice Lusa}
\coorientador{Paulo Henrique}
\instituicao{%
  Faculdades da Indústria -- IEL
  \par
  Sistemas de Informação} % Nome do curso
\tipotrabalho{Monografia (graduação)}
% O preambulo deve conter o tipo do trabalho, o objetivo, 
% o nome da instituição e a área de concentração 
\preambulo{Trabalho de Conclusão da Disciplina Jornada de Aprendizagem
 do período 7 apresentado à Faculdade da 
 Indústria de São José dos Pinhais, como requisito parcial 
 para obtenção do Título de Bacharel do Curso de Sistemas da Informação.}
% ---
\include{prétextuais/inalteraveis/pdf}
\include{misc}

% ----
% Início do documento
% ----
\begin{document}

% Seleciona o idioma do documento (conforme pacotes do babel)
%\selectlanguage{english}
\selectlanguage{brazil}

% Retira espaço extra obsoleto entre as frases.
\frenchspacing 

% ----------------------------------------------------------
% ELEMENTOS PRÉ-TEXTUAIS
% ----------------------------------------------------------

\imprimircapa

\imprimirfolhaderosto*

%\include{prétextuais/alteraveis/fichacatalografica}
\include{prétextuais/alteraveis/aprovacao}
\include{prétextuais/alteraveis/dedicatoria}
\include{prétextuais/alteraveis/agradecimentos}
\include{prétextuais/alteraveis/epigrafe}
\include{prétextuais/alteraveis/resumo}
\include{prétextuais/inalteraveis/figuras}
\include{prétextuais/inalteraveis/quadros}
\include{prétextuais/inalteraveis/tabelas}
\include{prétextuais/alteraveis/siglas}
\include{prétextuais/alteraveis/simbolos}
\include{prétextuais/inalteraveis/sumario}

% ----------------------------------------------------------
% ELEMENTOS TEXTUAIS
% ----------------------------------------------------------
\textual

\chapter{Introdução}

Este documento e seu código-fonte são exemplos de referência de uso da classe
\textsf{abntex2} e do pacote \textsf{abntex2cite}. O documento 
exemplifica a elaboração de trabalho acadêmico (tese, dissertação e outros do
gênero) produzido conforme a ABNT NBR 14724:2011 \emph{Informação e documentação
- Trabalhos acadêmicos - Apresentação}.

A expressão ``Modelo Canônico'' é utilizada para indicar que \abnTeX\ não é
modelo específico de nenhuma universidade ou instituição, mas que implementa tão
somente os requisitos das normas da ABNT. Uma lista completa das normas
observadas pelo \abnTeX\ é apresentada em \citeonline{abntex2classe}.

Sinta-se convidado a participar do projeto \abnTeX! Acesse o site do projeto em
\url{http://www.abntex.net.br/}. Também fique livre para conhecer,
estudar, alterar e redistribuir o trabalho do \abnTeX, desde que os arquivos
modificados tenham seus nomes alterados e que os créditos sejam dados aos
autores originais, nos termos da ``The \LaTeX\ Project Public
License.

Encorajamos que sejam realizadas customizações específicas deste exemplo para
universidades e outras instituições --- como capas, folha de aprovação, etc.
Porém, recomendamos que ao invés de se alterar diretamente os arquivos do
\abnTeX, distribua-se arquivos com as respectivas customizações.
Isso permite que futuras versões do \abnTeX~não se tornem automaticamente
incompatíveis com as customizações promovidas. Consulte
\citeonline{abntex2-wiki-como-customizar} para mais informações.

Este documento deve ser utilizado como complemento dos manuais do \abnTeX\ 
\cite{abntex2classe,abntex2cite,abntex2cite-alf} e da classe \textsf{memoir}
\cite{memoir}. 

Esperamos, sinceramente, que o \abnTeX\ aprimore a qualidade do trabalho que
você produzirá, de modo que o principal esforço seja concentrado no principal:
na contribuição científica.
Equipe \abnTeX 

Lauro César Araujo

\chapter{modelo de trabalho acadêmico}

\section{Quadros}

Este modelo vem com o ambiente \texttt{quadro} e impressão de Lista de quadros 
configurados por padrão. Verifique o \autoref{quadro_exemplo}.

\begin{quadro}[htb]
\caption{\label{quadro_exemplo}Exemplo de quadro}
\begin{tabular}{|c|c|c|c|}
	\hline
	\textbf{Pessoa} & \textbf{Idade} & \textbf{Peso} & \textbf{Altura} \\ \hline
	Marcos & 26    & 68   & 178    \\ \hline
	Ivone  & 22    & 57   & 162    \\ \hline
	...    & ...   & ...  & ...    \\ \hline
	Sueli  & 40    & 65   & 153    \\ \hline
\end{tabular}
\fonte{Autor.}
\end{quadro}

Este parágrafo apresenta como referenciar o quadro no texto, requisito
obrigatório da ABNT. 
Primeira opção, utilizando \texttt{autoref}: Ver o \autoref{quadro_exemplo}. 
Segunda opção, utilizando  \texttt{ref}: Ver o Quadro \ref{quadro_exemplo}.

\section{Figuras}

Veja a \autoref{logo_iel}.

\begin{figure}[!htb]
	\centering
	\caption{\label{logo_iel}Logo Faculdades da Indústria}
	\includegraphics[scale = 0.5]{images/Faculdade-da-Indústria-IEL-25.png}
	\fonte{\citeonline{abntex2-wiki-como-customizar}}
\end{figure}

\chapter{Tabelas}

\section{Tabelas}

Exemplo de inserção de tabela:

\begin{table}[!htb]
	\centering
	\caption{Alguns Estados e Capitais do Brasil}
	\begin{tabular}{ll}
	\rowcolor[HTML]{CBCEFB}
	\multicolumn{1}{c}{\cellcolor[HTML]{CBCEFB}Estado} & Cidade         \\
	Paraná                                             & Curitiba       \\
	Fortaleza                                          & Ceará          \\
	Minas Gerais                                       & Belo Horizonte
	\end{tabular}
	\fonte{Autor.}
\end{table}

\chapter{Seções}

\section{Exemplo de Seção}

Conteúdo da Seção.

\subsection{Exemplo de subseção}

Conteúdo da subseção.

\subsubsection{Exemplo de subsubseção}

Conteúdo da subsubseção.

\chapter{Referências}

Para referenciarmos um site podemos utilizar a tag \textbf{citeonline},
para livros utilizamos a tag \textbf{cite}, referenciar uma imagem, tabela,
quadro utilizamos \textbf{autoref}.

As referências são centralizadas no arquivo \textbf{referencias.bib} 
que possui uma estrutura simples e de fácil entendimento. 
Você pode utilizar alguns softwares, como o Zotero, para gerenciar suas bibliografias
e que exportam o arquivo com extensão .bib com todas suas referências. 
Caso não queira utilizar um software, também é fácil editar manualmente.

\chapter{Conclusão}

\lipsum[31-33]

% ----------------------------------------------------------
% ELEMENTOS PÓS-TEXTUAIS
% ----------------------------------------------------------
\postextual
% ----------------------------------------------------------

% ----------------------------------------------------------
% Referências bibliográficas
% ----------------------------------------------------------
\renewcommand{\bibname}{REFER\^ENCIAS}
\bibliography{referencias}

% ----------------------------------------------------------
% Glossário
% ----------------------------------------------------------
\include{póstextuais/alteraveis/apendices}
\include{póstextuais/alteraveis/anexos}

%---------------------------------------------------------------------
% ÍNDICE REMISSIVO
%---------------------------------------------------------------------
\phantompart
\index{Exemplo de Índice}
\renewcommand\indexname{ÍNDICE}
\printindex
%---------------------------------------------------------------------

\end{document}
